%%
%% This is file `sample-manuscript.tex',
%% generated with the docstrip utility.
%%
%% The original source files were:
%%
%% samples.dtx  (with options: `manuscript')
%% 
%% IMPORTANT NOTICE:
%% 
%% For the copyright see the source file.
%% 
%% Any modified versions of this file must be renamed
%% with new filenames distinct from sample-manuscript.tex.
%% 
%% For distribution of the original source see the terms
%% for copying and modification in the file samples.dtx.
%% 
%% This generated file may be distributed as long as the
%% original source files, as listed above, are part of the
%% same distribution. (The sources need not necessarily be
%% in the same archive or directory.)
%%
%% The first command in your LaTeX source must be the \documentclass command.
\documentclass[manuscript,screen]{acmart}
\usepackage{bm}

% Paolo's definition. Yes, I have my way to define stuff. Please take
% a look
\input{mydef}


% Always include hyperref last
\usepackage[bookmarks=true,breaklinks=true,letterpaper=true,colorlinks,linkcolor=black,citecolor=blue,urlcolor=black]{hyperref}

%%
%% \BibTeX command to typeset BibTeX logo in the docs
\AtBeginDocument{%
  \providecommand\BibTeX{{%
    \normalfont B\kern-0.5em{\scshape i\kern-0.25em b}\kern-0.8em\TeX}}}

%% Rights management information.  This information is sent to you
%% when you complete the rights form.  These commands have SAMPLE
%% values in them; it is your responsibility as an author to replace
%% the commands and values with those provided to you when you
%% complete the rights form.
\setcopyright{acmcopyright}
\copyrightyear{2020}
\acmYear{2020}
\acmDOI{}

%% These commands are for a PROCEEDINGS abstract or paper.
%\acmConference[Woodstock '18]{Woodstock '18: ACM Symposium on Neural
%  Gaze Detection}{June 03--05, 2018}{Woodstock, NY}
%\acmBooktitle{Woodstock '18: ACM Symposium on Neural Gaze Detection,
%  June 03--05, 2018, Woodstock, NY}
%\acmPrice{15.00}
%\acmISBN{978-1-4503-XXXX-X/18/06}


%%
%% Submission ID.
%% Use this when submitting an article to a sponsored event. You'll
%% receive a unique submission ID from the organizers
%% of the event, and this ID should be used as the parameter to this command.
%%\acmSubmissionID{123-A56-BU3}

%%
%% The majority of ACM publications use numbered citations and
%% references.  The command \citestyle{authoryear} switches to the
%% "author year" style.
%%
%% If you are preparing content for an event
%% sponsored by ACM SIGGRAPH, you must use the "author year" style of
%% citations and references.
%% Uncommenting
%% the next command will enable that style.
%%\citestyle{acmauthoryear}

%%
%% end of the preamble, start of the body of the document source.
\begin{document}

%%%%%%%%%%%---SETME-----%%%%%%%%%%%%%
\title{Randomization and Sparse Matrix by Vector Multiplication }

\author{Abhishek Jain}
%\email{---}
\author{Ismail Bustany}
%\email{---}
\author{Paolo D'Alberto}
%\email{---}

%\affiliation{%
%  \institution{1 * -}
%  \streetaddress{2 * -}
%  \city{3 * -}
%  \state{4 * -}
%  \postcode{5 * -}
%}

\renewcommand{\shortauthors}{Jain et al.}

\begin{abstract}
A sparse matrix by vector multiplication (SpMV) is simplified by the
matrix non-zero elements and how they are stored. From these two,
there are different strategies to achieve the best computation. The
applications of SpMV are many, the way to store the matrix are many,
and the algorithms are many. However, there is no optimality without
considering the architecture: for example, the CPU is only one among
... many; here we shall present the effects of choosing among CPU,
GPU, and FPGAs.

When guided by proper principles, we can randomize by hardware
properly. By nature, randomization is resilient to counter techniques,
thus suitable to avoid worst case scenarios and improve performance on
average; however, it can nudge off an otherwise good solution. Like
preconditioning, randomization is advantageous when the matrix is
reused or it is a constant such as in the power method for page
ranking, Krilov's space, or convolutions for image classifications.
Randomization is also an optimization that any architecture may take
advantage although in different ways. Randomization is beneficial in
specific cases for general purpose architectures and it has a larger
application to custom architecture.

We shall show there are cases where we can improve by 15\% performance
for general purpose architectures and by 8x for custom architectures.

\end{abstract}

\maketitle

\section{Introduction}
The idea should be simple to state: Sparse computations have few
operations but they still have patterns and shapes. To achieve peak
performance is to exploit these patterns. In the same fashion,
specific patterns may reach catastrophic performance.  We are in the
business in seeking these disagreeable patterns and destroy them.



\section{Entropy}
Patterns in sparse matrices are often visually pleasing. We want
Entropy as a quantitative measure representing matrices patterns. We
introduce such a measure to summarize the sparsity and to guide
randomization heuristics.


\section{Uniform distribution}
If we are trying to split a SpMV into two computations with equal work
load, and then recursively split further, then we must have a uniform
distribution of non-zeros so that we can achieve a balanced work
distribution or a quick way to estimate such balance distribution.


\section{Workload and Parallelism}
What we do not want to have is a partition into lots of smaller
computations with very different workloads. Here we may want to
indicate cases where we could split the matrix into sparse and dense
formats.

\section{Guiding randomization}
There is random and there is Random.


\section{Call for a different strategy}
We want to find out randomization techniques that are suitable for
custom hardware but also what are the most common and simple
heuristics that can justified for any hardware.


\section{Experimental Results}

Plots and pots.


%\input{conclusion.tex}

%%%%%%%%% -- BIB STYLE AND FILE -- %%%%%%%%
\bibliographystyle{ACM-Reference-Format} \bibliography{ref}
%%%%%%%%%%%%%%%%%%%%%%%%%%%%%%%%%%%%

%\appendix{Review and Response}
%\input{review.tex}
\end{document}
